\documentclass{ruthesis}
\special{papersize=8.5in,11in}
\usepackage{csquotes}

\begin{document}

\phd  
\copyrightpage % you may remove however it must be included if you choose to register your copyright

\title{May the Force Be with You}  % your dissertation title
\author{Anakin Skywalker}  % your name
\program{Statistics}  % your academic program
\director{Obi-Wan Kenobi}  % name of your advisor
\approvals{4}  % number of defense committee members
\submissionyear{2019}   % year of graduation
\submissionmonth{OCTOBER}  % month of degree you are seeking. Only JANUARY, MAY, OCTOBER degree is offered as 10/2019. Capitalized Letters Needed!

\abstract{This is the abstract.
}

\beforepreface
\acknowledgements{Write down your acknowledgements.}
\dedication{Write down your dedication.}

\tablespage
\figurespage
\afterpreface

\noindent

\chapter{Introduction}
The Force is a metaphysical and ubiquitous power in the Star Wars universe. It is wielded by "Force-sensitive" heroes like the Jedi who seek to become one with the Force, while the Sith and other villains exploit the Force and have always tried to bend it towards their will. The Force has been compared to aspects of several world religions, and the phrase "May the Force be with you" has become part of the popular-culture vernacular.


\chapter{Development of the Force}
\section{Analysis of the Force}
\subsection{Religion and Spirituality}
In his 1977 review of Star Wars, Vincent Canby of The New York Times called the Force "a mixture of what appears to be ESP and early Christian faith." The Magic of Myth compares the sharp distinction between the good "light side" and evil "dark side" of the Force to Zoroastrianism, which posits that "good and evil, like light and darkness, are contrary realities". The connectedness between the light and dark sides has been compared to the relationship between yin and yang in Taoism, although the balance between yin and yang lacks the element of evil associated with the dark side. Taylor identifies other similarities between the Force and a Navajo prayer, prana, and qi. It is a common plot device in jidaigeki films like The Hidden Fortress (1958), which inspired Star Wars, for samurai who master qi to achieve astonishing feats of swordsmanship. Taylor added that the lack of detail about the Force makes it "a religion for the secular age". According to Jennifer Porter, professor of religious studies at the Memorial University of Newfoundland, "the Force is a metaphor for godhood that resonates and inspires within [people] a deeper commitment to the godhood identified within their traditional faith".

According to Christian Pastor Clayton Keenan, "the spirituality of 'Star Wars' has to do with the Force. It's depicted as ... something supernatural within this universe, but it's not the same thing as a personal god that Christians or Jews or Muslims might believe in. It's this impersonal force that is in some ways this neutral, impersonal energy that is out there to be used for good or for evil."

At one point, Francis Ford Coppola suggested to George Lucas that they use their combined fortunes to start a religion based on the Force. Practitioners of Jediism pray to and express gratitude to the Force.

\subsection{Scientific Perspectives}

Astrophysicist Jeanne Cavelos says in The Science of Star Wars that the Force raises questions that scientists have long asked. She points out that the ancient Greeks explored the idea of a "fifth element" that permeated and connected everything in the universe, and that Isaac Newton proposed that the human brain might be able to trigger waves in the ether, giving humans psychic powers. Cavelos's sources are mostly skeptical about a "real world" explanation for the Force, but they explore ideas in areas such as quantum physics, parapsychology, and the notion of science so advanced that it appears magical. Explaining the Force is particularly difficult, Cavelos says, because "it does so many different things". Several scientists have said it is best not to try to explain how the Force works.

Cavelos says the Force "suggests a universe quite different than the one we think we're living in", and that some unknown fields or particles might explain the Force. Cavelos believes vacuum energy is one option to power physical Force feats, and she says a fifth force beyond the four fundamental interactions might account for the Force. Flavio Fenton of the Georgia Institute of Technology School of Physics suggests a fifth force would carry two types of charge—one for the light side and one for the dark—and that each would be carried by its own particle. Nepomuk Otte, also from Georgia Tech, cautions that Newton's third law of motion says telekinesis would apply a force back on the Force-wielding character. Force powers like precognition are challenging to explain because of the implied time travel of information, but Cavelos suggests tachyons traveling faster than light might carry such information. Cavelos explores the possibility of brain implants or sensors being used to detect users' intent and manipulate energy fields to control the Force. Michio Kaku says Luke Skywalker's Jedi training on Dagobah might involve learning to control brain waves, and Cavelos compares such discipline to contemporary patients learning to control prosthetics.

\clearpage
\nocite{*}
\bibliographystyle{apalike}
\bibliography{dissertation}

\end{document}
